% Pengaturan ukuran teks dan bentuk halaman dua sisi
\documentclass[10pt,twoside]{report}

% Pengaturan ukuran halaman dan margin
\usepackage[a5paper,top=25mm,left=25mm,right=20mm,bottom=25mm]{geometry}

% Pengaturan ukuran spasi
\usepackage[singlespacing]{setspace}

% Judul dokumen
\title{Buku Laporan Kerja Praktik ITS}
\author{Musk, Elon Reeve \and Kjellberg, Felix Arvid Ulf}

% Pengaturan format bahasa
\usepackage[indonesian]{babel}

% Pengaturan detail pada file PDF
\usepackage[pdfauthor={\@author},bookmarksnumbered,pdfborder={0 0 0}]{hyperref}

% Pengaturan jenis karakter
\usepackage[utf8]{inputenc}

% Pengaturan ukuran indentasi paragraf
\setlength{\parindent}{2em}

% Pengaturan ukuran spasi paragraf
\setlength{\parskip}{0.5ex}

% Package lainnya
\usepackage{etoolbox} % Mengubah fungsi default
\usepackage{enumitem} % Pembuatan list
\usepackage{lipsum} % Pembuatan template kalimat
\usepackage{graphicx} % Input gambar
\usepackage{longtable} % Pembuatan tabel
\usepackage[table,xcdraw]{xcolor} % Pewarnaan tabel
\usepackage[numbers]{natbib} % Kutipan artikel
\usepackage{eso-pic} % Pembuatan background
\usepackage{changepage} % Pembuatan teks kolom
\usepackage{wrapfig} % Wrapping gambar

% Definisi untuk "Hati ini sengaja dikosongkan"
\def\kosong{
	\vspace*{\fill}
	\begin{center}\textit{Halaman ini sengaja dikosongkan}\end{center}
	\vfill
}
\patchcmd{\cleardoublepage}{\hbox{}}{\kosong}{}{}

% Pengaturan penomoran halaman
\usepackage{fancyhdr}
\fancyhf{}
\renewcommand{\headrulewidth}{0pt}
\pagestyle{fancy}
\fancyfoot[CE,CO]{\thepage}
\patchcmd{\chapter}{plain}{fancy}{}{}
\patchcmd{\chapter}{empty}{plain}{}{}

% Pengaturan format judul bab
\usepackage{titlesec}
\titleformat{\chapter}[display]{\bfseries\Large}{BAB \centering\Roman{chapter}}{0ex}{\vspace{0ex}\centering}[\vspace{2ex}]
\titleformat{\section}{\bfseries\large}{\MakeUppercase{\thesection}}{1ex}{}
\titleformat{\subsection}{\bfseries\large}{\MakeUppercase{\thesubsection}}{1ex}{}
\titleformat{\subsubsection}{\bfseries\large}{\MakeUppercase{\thesubsubsection}}{1ex}{}
\titlespacing{\chapter}{0ex}{0ex}{2ex}
\titlespacing{\section}{0ex}{2ex}{1ex}
\titlespacing{\subsection}{0ex}{1ex}{0.5ex}
\titlespacing{\subsubsection}{0ex}{1ex}{0.5ex}

% Pengaturan persamaan
\newenvironment{conditions}
{\par\vspace{\abovedisplayskip}\noindent
	\tabularx{\columnwidth}{>{$}l<{$} @{${}={}$} >{\raggedright\arraybackslash}X}}
{\endtabularx\par\vspace{\belowdisplayskip}}

% Pengaturan format baris program
\usepackage{listings}
\definecolor{comment}{RGB}{0,128,0}
\definecolor{string}{RGB}{255,0,0}
\definecolor{keyword}{RGB}{0,0,255}
\lstdefinestyle{codestyle}{
	commentstyle=\color{comment},
	stringstyle=\color{string},
	keywordstyle=\color{keyword},
	basicstyle=\footnotesize\ttfamily,
	numbers=left,
	numberstyle=\tiny,
	numbersep=5pt,
	frame=lines,
	breaklines=true,
	prebreak=\raisebox{0ex}[0ex][0ex]{\ensuremath{\hookleftarrow}},
	showstringspaces=false,
	upquote=true,
	tabsize=2,
}
\lstset{style=codestyle}

% Isi keseluruhan dokumen
\begin{document}

  % Nomor halaman pembuka dimulai dari sini
  \pagenumbering{roman}

  % Sampul luar
  \AddToShipoutPictureBG*{
  \AtPageLowerLeft{
    \hspace{-3.5mm}
    \raisebox{0mm}{
      \includegraphics[width=\paperwidth,height=\paperheight]{sampul/sampul-luar.png}
    }
  }
}

\thispagestyle{empty}

\newgeometry{top=70mm,left=25mm,right=20mm,bottom=25mm}

\begin{flushleft}

  \sffamily
  \color{white}

  \noindent
  \textbf{KERJA PRAKTIK - TD123456}
  \vspace{4ex}

  \noindent
  {\large \textbf{PT. NATIONAL AERONAUTICS AND SPACE ADMINISTRATION}} \\
  \textbf{(02 Maret 2020 s/d 02 April 2020)}
  \vspace{6ex}

  \noindent
  {\large \textbf{PEMBUATAN ROKET LUAR ANGKASA ANTI GRAVITASI UNTUK PT. NASA}}
  \vspace{6ex}

  \begin{adjustwidth}{-0.2cm}{}
    \begin{tabular}{lcp{0.7\linewidth}}
      \textbf{Elon Reeve Musk} & & \textbf{NRP 0123 20 4000 0001} \\
      \textbf{Felix Arvid Ulf Kjellberg} & & \textbf{NRP 0123 20 4000 0002} \\
    \end{tabular}
  \end{adjustwidth}
  \vspace{4ex}

  \noindent
  \textbf{Dosen Pembimbing} \\
  \textbf{Nikola Tesla, S.T., M.T.}
  \vspace{12ex}

  \noindent
  \textbf{DEPARTEMEN TEKNIK DIRGANTARA} \\
  \textbf{Fakultas Teknologi Dirgantara} \\
  \textbf{Institut Teknologi Sepuluh Nopember} \\
  \textbf{Surabaya 2020}

\end{flushleft}

\restoregeometry
  \cleardoublepage

  % Sampul dalam
  \input{sampul/sampul-dalam.tex}
  \cleardoublepage

  % Lembar pengesahaan untuk departemen
  \begin{center}
  {\Large \textbf{LEMBAR PENGESAHAN}}
  \vspace{6ex}

  \addcontentsline{toc}{chapter}{LEMBAR PENGESAHAN (DEPARTEMEN)}

  {\large \textbf{PEMBUATAN ROKET LUAR ANGKASA ANTI GRAVITASI UNTUK PT. NASA}}
  \vspace{6ex}

  Laporan Kerja Praktek ini disusun untuk \lipsum[1][1]
  \vspace{2ex}

  Tempat Pengesahan di: Surabaya \\
  Tanggal: 2 Maret 2020
  \vspace{8ex}

  Menyetujui, \\
  Dosen Pembimbing,
  \vspace{12ex}

  \textbf{\underline{Nikola Tesla, S.T., M.T.}} \\
  NIP. 18560710 194301 1 001
  \vspace{8ex}

  Mengetahui, \\
  Kepala Departemen Teknik Dirgantara FTD - ITS,
  \vspace{12ex}

  \textbf{\underline{Dr. Leonardo Da Vinci, S.T., M.T.}} \\
  NIP 14520415 151905 1 001

\end{center}
  \cleardoublepage

  % Lembar pengesahan untuk perusahaan
  \begin{center}
  {\Large \textbf{LEMBAR PENGESAHAN}}
  \vspace{6ex}

  \addcontentsline{toc}{chapter}{LEMBAR PENGESAHAN}

  {\large \textbf{PEMBUATAN ROKET LUAR ANGKASA ANTI GRAVITASI UNTUK PT. NASA}}
  \vspace{6ex}

  Laporan Kerja Praktek ini disusun untuk \lipsum[1][1]
  \vspace{2ex}

  Tempat Pengesahan di: California \\
  Tanggal: 2 Maret 2020
  \vspace{8ex}

  Mengetahui, \\
  Pembimbing Perusahaan
  \vspace{12ex}

  \textbf{\underline{Yuri Gagarin, S.Si., M.Si.,}}
  \vspace{8ex}

  Mengetahui, \\
  Chief Executive Officer PT. NASA
  \vspace{12ex}

  \textbf{\underline{Dr. Galileo Galilei, S.Si., M.Si.}}

\end{center}
  \cleardoublepage

  % Kata pengantar
  \begin{center}
  \Large\textbf{KATA PENGANTAR}
\end{center}
\vspace{1ex}

\addcontentsline{toc}{chapter}{KATA PENGANTAR}

\setlength{\parindent}{7ex}

Puji dan syukur kehadirat \lipsum[1][1-5]
\vspace{0.5ex}

Penelitian ini disusun dalam rangka \lipsum[1][1-5]
Oleh karena itu, penulis mengucapkan terima kasih kepada:
\vspace{0.5ex}

\begin{enumerate}[nolistsep]

  \item Keluarga, Ibu, Bapak dan Saudara tercinta yang telah \lipsum[1][1]
  \vspace{0.5ex}

  \item Bapak Nikola Tesla, S.T., M.T., selaku \lipsum[1][1-2]
  \vspace{0.5ex}

  \item \lipsum[1][1-3]
  \vspace{0.5ex}

\end{enumerate}
\vspace{0.5ex}

Akhir kata, semoga \lipsum[1][1-5]
\vspace{2ex}

\begin{flushright}
  \begin{tabular}[b]{c}
    Surabaya, Maret 2020
    \\
    \\
    \\
    \\
    Penulis
  \end{tabular}
\end{flushright}
  \cleardoublepage

  % Daftar isi
  \renewcommand*\contentsname{DAFTAR ISI}
  \addcontentsline{toc}{chapter}{\contentsname}
  \tableofcontents
  \cleardoublepage

  % Daftar gambar
  \renewcommand*\listfigurename{DAFTAR GAMBAR}
  \addcontentsline{toc}{chapter}{\listfigurename}
  \listoffigures
  \cleardoublepage

  % Daftar tabel
  \renewcommand*\listtablename{DAFTAR TABEL}
  \addcontentsline{toc}{chapter}{\listtablename}
  \listoftables
  \cleardoublepage

  % Nomor halaman isi dimulai dari sini
  \pagenumbering{arabic}

  % Bab 1 pendahuluan
	% Ubah kalimat sesuai dengan judul dari bab ini
\chapter{PENDAHULUAN}

% Ubah konten-konten berikut sesuai dengan yang ingin diisi pada bab ini

\section{Latar Belakang}

Pesatnya perkembangan roket yang merupakan \lipsum[1][1-15]

\lipsum[2][1-10]

\section{Rumusan Permasalahan}

Masalah yang akan \lipsum[3][1-2] adalah:

\begin{enumerate}[nolistsep]

  \item Bagaimana cara \lipsum[3][3-5]

  \item \lipsum[3][6-8]

\end{enumerate}

\section{Tujuan}

Tujuan dari \lipsum[4][1-3] adalah:

\begin{enumerate}[nolistsep]

  \item Membuat \lipsum[4][4-5]

  \item \lipsum[4][6-9]

\end{enumerate}

\section{Manfaat}

Manfaat dari \lipsum[5][1-3] adalah:

\begin{enumerate}[nolistsep]

  \item Mempermudah \lipsum[5][4-5]

  \item \lipsum[5][6-10]

\end{enumerate}

\section{Waktu dan Tempat Pelaksanaan}

Kerja praktik akan dilaksanakan pada \lipsum[6][1-3]

\section{Metodologi Kerja Praktik}

Metode yang \lipsum[7][1-5] yaitu:

\begin{enumerate}[nolistsep]

  \item \textbf{Perumusan Masalah}

  Pada tahap ini \lipsum[7][6-9]

  \item \textbf{Studi Literatur}

  Pada tahap ini \lipsum[7][10-13]

  \item \textbf{Analisis dan Perancangan Sistem}

  Pada tahap ini \lipsum[8][1-2]

  \item \textbf{Implementasi Sistem}

  Pada tahap ini \lipsum[8][3-6]

  \item \textbf{Pengujian dan Evaluasi}

  Pada tahap ini \lipsum[8][7-12]

\end{enumerate}

\section{Sistematika Penulisan}

Laporan kerja praktik akan terbagi menjadi \lipsum[9][1] yaitu:

\begin{enumerate}[nolistsep]

  \item \textbf{Bab I Pendahuluan}

  Bab ini berisi \lipsum[9][2-4]

  \item \textbf{Bab II Profil Perusahaan}

  Bab ini berisi \lipsum[9][5-7]

  \item \textbf{Bab III Tinjauan Pustaka}

  Bab ini berisi \lipsum[9][8]

  \item \textbf{Bab IV Desain dan Implementasi}

  Bab ini berisi \lipsum[10][1-2]

  \item \textbf{Bab V Pengujian dan Evaluasi}

  Bab ini berisi \lipsum[10][3-4]

  \item \textbf{Bab VI Kesimpulan dan Saran}

  Bab ini berisi \lipsum[10][5-8]

\end{enumerate}

  \cleardoublepage

  % Bab 2 profil perusahaan
	% Ubah kalimat sesuai dengan judul dari bab ini
\chapter{PROFIL PERUSAHAAN}

% Ubah konten-konten berikut sesuai dengan yang ingin diisi pada bab ini

\section{Sejarah PT. NASA}

PT. NASA berdiri pada \lipsum[11]

\lipsum[12][1-10]

\section{Visi dan Misi}

PT. NASA memiliki \lipsum[13][1-3] sebagai berikut:

\begin{enumerate}[nolistsep]

  \item \textbf{Visi PT. NASA}

  Menjadi \lipsum[13][4-7]

  \item \textbf{Misi PT. NASA}

  \begin{enumerate}[nolistsep]

    \item Membuat \lipsum[13][8-9]

    \item \lipsum[13][10-12]

  \end{enumerate}

\end{enumerate}

\section{Struktur Organisasi}

Struktur Organisasi dari \lipsum[14][1-8]

% Contoh input gambar dengan format *.png
\begin{figure} [ht] \centering
  % Nama dari file gambar yang diinputkan
  \includegraphics[scale=0.4]{gambar/organization-structure.png}
  % Keterangan gambar yang diinputkan
  \caption{Struktur organisasi PT. NASA}
  % Label referensi dari gambar yang diinputkan
	\label{fig:OrganizationStructure}
\end{figure}

% Contoh penggunaan referensi dari gambar yang diinputkan
Seperti yang bisa dilihat pada Gambar \ref{fig:OrganizationStructure}, \lipsum[15]

  \cleardoublepage

  % Bab 3 tunjauan pustaka
	% Ubah kalimat sesuai dengan judul dari bab ini
\chapter{TINJAUAN PUSTAKA}

% Ubah konten-konten berikut sesuai dengan yang ingin diisi pada bab ini

\section{Roket Luar Angkasa}

% Contoh input gambar dengan format *.jpg
\begin{figure} [ht] \centering
  % Nama dari file gambar yang diinputkan
	\includegraphics[scale=0.3]{gambar/space-shuttle.jpg}
  % Keterangan gambar yang diinputkan
	\caption{Peluncuran pesawat luar angkasa Discovery \citep{DiscoverySpaceShuttle}}
  % Label referensi dari gambar yang diinputkan
	\label{fig:SpaceShuttle}
\end{figure}

Roket luar angkasa merupakan \lipsum[16][1-10]

% Contoh penggunaan referensi dari gambar yang diinputkan
\emph{Discovery}, Gambar \ref{fig:SpaceShuttle}, merupakan \lipsum[17][1-9]

\section{Gravitasi}

Gravitasi merupakan \lipsum[18][1-10]

\subsection{Hukum Newton}

% Contoh penggunaan referensi dari pustaka
Newton \citep{Newton1687} pernah merumuskan bahwa \lipsum[19]
% Contoh penggunaan referensi dari persamaan
Kemudian menjadi persamaan seperti pada persamaan \ref{eq:FirstNewtonLaw}.

% Contoh pembuatan persamaan
\begin{equation}
  % Label referensi dari persamaan yang dibuat
  \label{eq:FirstNewtonLaw}
  % Baris kode persamaan yang dibuat
  \sum \mathbf{F} = 0\; \Leftrightarrow\; \frac{\mathrm{d} \mathbf{v} }{\mathrm{d}t} = 0.
\end{equation}

\subsection{Anti Gravitasi}

Anti gravitasi merupakan \lipsum[20]

  \cleardoublepage

  % Bab 4 desain dan implementasi
	% Ubah kalimat sesuai dengan judul dari bab ini
\chapter{DESAIN DAN IMPLEMENTASI}

% Ubah konten-konten berikut sesuai dengan yang ingin diisi pada bab ini

\section{Deskripsi Sistem}

Sistem akan dibuat dengan \lipsum[21][1-12]

\section{Implementasi Alat}

Alat diimplementasikan dengan \lipsum[22]

% Contoh pembuatan code snippet
\begin{lstlisting}[
  language=C++,
  label={lst:Hello World},
  caption={Program hello world}
]
#include <iostream>

int main() {
    std::cout << "Hello World!";
    return 0;
}
\end{lstlisting}

% Contoh penggunaan referensi dari code snippet yang diinputkan
Seperti contoh pada baris program Listing \ref{lst:Hello World} dan Listing \ref{lst:PrimeNumber}, \lipsum[23]

% Contoh input code snippet
\lstinputlisting[
  % Bahasa yang digunakan oleh code snippet
  language=Python,
  % Label referensi dari code snippet yang diinputkan
  label={lst:PrimeNumber},
  % Keterangan dari code snippet yang diinputkan
  caption={Program perhitungan bilangan prima}
% Nama dari file code snippet yang diinputkan
]{program/prime-number.py}

  \cleardoublepage

  % Bab 5 pengujian dan evaluasi
	% Ubah kalimat sesuai dengan judul dari bab ini
\chapter{PENGUJIAN DAN EVALUASI}

% Ubah konten-konten berikut sesuai dengan yang ingin diisi pada bab ini

\section{Skenario Pengujian}

Pengujian dilakukan dengan \lipsum[24]

\section{Evaluasi Pengujian}

Dari pengujian yang \lipsum[25][1-10]

% Contoh input konten dari file terpisah
% Contoh pembuatan tabel
\begin{longtable}{|l|l|}
  % Keterangan dari tabel yang dibuat
	\caption{Hasil Pengukuran Energi dan Kecepatan}
  % Label referensi dari tabel yang dibuat
	\label{tb:energiKecepatan}\\
  % Isi dari tabel yang dibuat
  \hline
  \rowcolor[HTML]{C0C0C0}
  \textbf{Energi} & \textbf{Kecepatan} \\ \hline
	10 J & 200 m/s \\ \hline
	20 J & 400 m/s \\ \hline
	30 J & 800 m/s \\ \hline
	40 J & 1600 m/s \\ \hline
\end{longtable}

% Contoh penggunaan referensi dari tabel yang dibuat
Sesuai dengan hasil pada Tabel \ref{tb:EnergiKecepatan}, didapatkan bahwa energi yang \lipsum[26]

  \cleardoublepage

  % Bab 6 kesimpulan dan saran
	% Ubah kalimat sesuai dengan judul dari bab ini
\chapter{KESIMPULAN DAN SARAN}

% Ubah konten-konten berikut sesuai dengan yang ingin diisi pada bab ini

\section{Kesimpulan}

Kesimpulan yang kami peroleh dari \lipsum[28][1-3] adalah:

\begin{enumerate}[nolistsep]

  \item Pembuatan \lipsum[29][1-3]

  \item \lipsum[29][4-6]

  \item \lipsum[29][7-9]

\end{enumerate}

\section{Saran}

Saran yang kami ajukan dalam \lipsum[30][1-2] antara lain:

\begin{enumerate}[nolistsep]

  \item Sebaiknya \lipsum[31][1-3]

  \item \lipsum[31][4-6]

  \item \lipsum[31][7-9]

\end{enumerate}

  \cleardoublepage

  % Daftar pustaka
  \renewcommand\bibname{DAFTAR PUSTAKA}
  \addcontentsline{toc}{chapter}{\bibname}
  \bibliographystyle{unsrtnat}
  \bibliography{pustaka/pustaka.bib}
  \cleardoublepage

  % Biografi penulis
	\begin{center}
  \Large\textbf{BIOGRAFI PENULIS}
\end{center}
\vspace{2ex}

\addcontentsline{toc}{chapter}{BIOGRAFI PENULIS}

\begin{wrapfigure}{L}{0.3\textwidth}
	\centering
	\vspace{-3ex}
	\includegraphics[width=0.3\textwidth]{gambar/elon.jpg}
	\vspace{-4ex}
\end{wrapfigure}

\noindent Elon Reeve Musk, lahir pada \lipsum[1]
\vspace{0.5ex}

\vspace{2ex}

\begin{wrapfigure}{L}{0.3\textwidth}
	\centering
	\vspace{-3ex}
	\includegraphics[width=0.3\textwidth]{gambar/felix.jpg}
	\vspace{-4ex}
\end{wrapfigure}

\noindent Felix Arvid Ulf Kjellberg, lahir pada \lipsum[2]
\vspace{0.5ex}
  \cleardoublepage

\end{document}