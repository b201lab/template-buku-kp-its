\chapter{PENGUJIAN DAN EVALUASI}
\vspace{4ex}

\setlength{\parindent}{7ex}

\section{Skenario Pengujian}
\vspace{1ex}

Pengujian dilakukan dengan \lipsum[1]
\vspace{0.5ex}

\lipsum[2]
\vspace{0.5ex}

\newpage

\section{Evaluasi Pengujian}
\vspace{1ex}

Dari pengujian yang \lipsum[3]
\vspace{0.5ex}

Sesuai dengan hasil pada \ref{tb:energi_kecepatan} didapatkan \lipsum[4]

\begin{longtable}{|l|l|}
	\caption{Hasil Pengukuran Energi dan Kecepatan}
	\vspace{1.5ex}
	\label{tb:energi_kecepatan}\\
	\hline
	\rowcolor[HTML]{C0C0C0}
	\textbf{Energi} & \textbf{Kecepatan} \\ \hline
	10 J & 200 m/s \\ \hline
	20 J & 400 m/s \\ \hline
	30 J & 800 m/s \\ \hline
	40 J & 1600 m/s \\ \hline
\end{longtable}
\vspace{1ex}

\lipsum[5]
\vspace{0.5ex}