% Ubah kalimat sesuai dengan judul dari bab ini
\chapter{DESAIN DAN IMPLEMENTASI}
\vspace{4ex}

% Pengaturan ukuran indentasi
\setlength{\parindent}{7ex}

% Ubah konten-konten berikut sesuai dengan yang ingin diisi pada bab ini

\section{Deskripsi Sistem}
\vspace{1ex}

Sistem akan dibuat dengan \lipsum[1]
\vspace{0.5ex}

\section{Implementasi Alat}
\vspace{1ex}

Alat diimplementasikan dengan \lipsum[2]
\vspace{0.5ex}

% Digunakan untuk page break
\newpage

% Contoh pembuatan code snippet
\begin{lstlisting}[
  language=C++,
  label={lst:helloWorld},
  caption={Hello World}
]
#include <iostream>

int main() {
    std::cout << "Hello World!";
    return 0;
}
\end{lstlisting}
\vspace{0.5ex}

% Contoh penggunaan referensi dari code snippet yang diinputkan
Seperti contoh pada baris program \ref{lst:helloWorld} dan \ref{lst:bilanganPrima}, \lipsum[3]
\vspace{0.5ex}

% Contoh input code snippet
\lstinputlisting[
  % Bahasa yang digunakan oleh code snippet
  language=Python,
  % Label referensi dari code snippet yang diinputkan
  label={lst:bilanganPrima},
  % Keterangan dari code snippet yang diinputkan
  caption={Perhitungan Bilangan Prima}
% Nama dari file code snippet yang diinputkan
]{program/prime-number.py}
\vspace{0.5ex}