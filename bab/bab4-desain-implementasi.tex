\chapter{DESAIN DAN IMPLEMENTASI}
\vspace{4ex}

\setlength{\parindent}{7ex}

\section{Deskripsi Sistem}
\vspace{1ex}

Sistem akan dibuat dengan \lipsum[1]
\vspace{0.5ex}

\lipsum[2]
\vspace{0.5ex}

\newpage

\section{Implementasi Alat}
\vspace{1ex}

Alat diimplementasikan dengan \lipsum[3]
\vspace{0.5ex}

\begin{lstlisting}[
  language=Python,
  label={lst:bilanganPrima},
  caption={Perhitungan Bilangan Prima}
]
def isPrimeNumber(num):
    # prime numbers are greater than 1
    if num > 1:
        # check for factors
        for i in range(2,num):
            if (num % i) == 0:
                return False
        else:
            return True
    # if input number is less than
    # or equal to 1, it is not prime
    else:
        return False
\end{lstlisting}

Seperti contoh pada baris program \ref{lst:bilanganPrima} \lipsum[4]
\vspace{0.5ex}